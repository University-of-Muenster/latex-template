\appendix
\AnhChapter{Allgemeine Erl�uterungen des Anhangs}
Weitere Informationen werden im Anhang aufgef�hrt (\zB Listings). F�r die Untergliederung des Anhangs wird die FV \glqq �berschrift 8\grqq\ benutzt.
Sollte eine tiefere Untergliederung notwendig sein, so kann die FV \glqq �berschrift 9\grqq\ verwendet werden. Je nach Bedeutung und Umfang der 
Anhangskapitel kann und sollte der Anhang auch in das Inhaltsverzeichnis aufgenommen werden. Hierbei ist Folgendes zu beachten.
\begin{itemize}
	\item In diesem Beispieldokument ist der Anhang mit erster und zweiter Gliederungsstufe im Inhaltsverzeichnis aufgef�hrt. Um eine oder beide 
			Gliederungsebenen des Anhangs aus dem Inhaltsverzeichnis zu entfernen, empfiehlt sich folgende Vorgehensweise: Zun�chst wird das 
			Inhaltsverzeichnis angesprungen. Dann kann im Men� \glqq Einf�gen\grqq\ der Punkt \glqq Index und Verzeichnisse\dots\grqq\ ausgew�hlt werden.
			Im Karteireiter \glqq Inhaltsverzeichnis\grqq\ k�nnen dann unter \glqq Optionen\grqq\ die Inhaltsverzeichnisebenen f�r die Formatvorlagen
			Verzeichnis8 und Verzeichnis9 (voreingestellt sind hier die Werte 2 und 3) entfernt werden.
	\item Die Erw�hnung einzelner Abschnitte des Anhangs im Inhaltsverzeichnis dient lediglich der �bersichtlichkeit. Dies ist wiederum nur dann 
			erforderlich, wenn der Anhang einen entsprechenden Umfang aufweist. Ein insgesamt dreiseitiger Anhang erf�llt dieses Kriterium \bspw 
			nicht.\footnote{Der hier vorliegende Anhang erf�llt das Kriterium ebenfalls nicht und ist lediglich zu Demonstrationszwecken in kompletter
			Gliederung in das Inhaltsverzeichnis integriert.}
	\item Die Gliederung des Anhangs darf nur einen Bruchteil des Inhaltsverzeichnisses darstellen. Ansonsten stellt sich automatisch die Frage, 
			weshalb der Anhang bei einem solchen Umfang und vorliegender Struktur nicht direkt in die Arbeit eingegliedert worden ist.
	\item Es sollten nur wirklich entscheidende und grunds�tzlich unterschiedliche Abschnitte des Anhangs auf separate Gliederungspunkte verteilt 
			werden.
	\item Der Anhang geht zwar in die Bewertung der Arbeit ein, allerdings muss die Arbeit auch ohne Kenntnis des Anhangs zu verstehen sein.
\end{itemize}

\AnhChapter{Beispiel 1: BASIC-Listing Modul \glqq Sales and Distribution SD\grqq }
\begin{BASIC}
10 PRINT "Sales and Distribution"
20 GOTO 10
\end{BASIC}

\AnhChapter{Beispiel 2: Abbildungen und Tabellen im Anhang}
\begin{itemize}
	\item Abbildungen sowie Tabellen werden im Anhang mit der Aufz�hlung \glqq Anh.\grqq\ versehen. Die Beschriftung erfolgt in beiden F�llen �ber das
			Makro \glqq AnhangsbeschriftungEinf�gen\grqq
	\item Abbildungen und Tabellen des Anhangs werden nicht im Abbildungs- \resp Tabellenverzeichnis aufgef�hrt.
	\item Auch im Anhang ist auf korrekte Referenzierung der Quellen zu achten.
\end{itemize}

